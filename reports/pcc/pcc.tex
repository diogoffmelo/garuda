\documentclass[openany, a4paper,12pt, oneside]{article}
\usepackage[utf8]{inputenc}
\usepackage[brazil]{babel}
\usepackage{amsfonts}
\usepackage{multirow}
\usepackage{amssymb}
\usepackage{graphicx}
\usepackage{subfigure}
\usepackage{units}
\usepackage{placeins}
\usepackage{listings}
\usepackage{tabularx,colortbl}
\usepackage{fancyhdr,lastpage}
\usepackage{color}
\usepackage{algorithm}
\usepackage{algpseudocode}
\usepackage{anysize}
\marginsize{3.0cm}{2.5cm}{3.0cm}{2.5cm}
\oddsidemargin 0.0cm
\usepackage{transparent}
\usepackage{eso-pic}
\usepackage[T1]{fontenc}
\usepackage{lscape}
\usepackage{lettrine}
\usepackage{etoolbox}

\usepackage{diagbox}
%\usepackage{slashbox}


\apptocmd{\thebibliography}{\csname phantomsection\endcsname\addcontentsline{toc}{chapter}{\bibname}}{}{}

\newcolumntype{L}[1]{>{\raggedright\let\newline\\\arraybackslash\hspace{0pt}}m{#1}}
\newcolumntype{C}[1]{>{\centering\let\newline\\\arraybackslash\hspace{0pt}}m{#1}}
\newcolumntype{R}[1]{>{\raggedleft\let\newline\\\arraybackslash\hspace{0pt}}m{#1}}
\usepackage{rotating}
\usepackage{setspace}
\usepackage[titletoc]{appendix}
\usepackage{pdfpages}


\begin{document}
\pagestyle{empty}
\begin{flushright}
	\noindent\rule{15cm}{0.4pt}\\[0.5cm]
	\textbf{{ \LARGE Projeto de Conclusão de Curso}}\\[0.1cm]
	\noindent\rule{15cm}{0.4pt}\\[7cm]
	\textbf{{\Large Aprendizado profundo com capacidade computacional reduzida: uma aplicação à quebra de captchas.}}\\[4cm]
\end{flushright}

\begin{center}	
	\textbf{\large Diogo Felipe Félix de Melo}\\[3cm]	
	
	\textbf{\large Área de Concentra\c{c}\~{a}o:} Aprendizado de Máquina.\\	
	\textbf{\large Orientador(a):} Pablo de Azevedo Sampaio\\[2cm]
	\vfill
	\textsc{Recife, Maio/2018}.
\end{center}
\pagebreak
\pagenumbering{gobble}

\begin{center}	
	\textbf{\textsc{\large Documento de Projeto de Pesquisa}}\\[1cm]
\end{center}
\section{Identificac\~{a}o}

\textbf{Aluno(a):} Diogo Felipe Félix de Melo (diogoffmelo@gmail.com)\\
\textbf{Orientador(a):} Pablo de Azevedo Sampaio (prof.pablo.sampaio@gmail.com)\\
\textbf{Título:} Aprendizado profundo com capacidade computacional reduzida: uma aplicação à quebra de captchas.\\
\textbf{Área de Concentração:} Aprendizado de Máquina\\
\textbf{Linha de Pesquisa:} Redes Neurais de aprendizado profundo\\


\section{Introdução}

Modelos de aprendizado baseados em neurologia são conhecidos desde meados do século passado\cite{perceptron_58}. Das proposições iniciais até os dias de hoje, essa classe modelos tem evoluído em complexidade e técnicas de forma contínua,
culminando em modelos com muitas camadas e níveis cada vez mais abstratos de representações (ver \cite{Goodfellow-et-al-2016} para uma breve revisão histórica).
Apesar dos avanços na área, foi apenas recentemente que modelos neurais 
começaram a redefinir o estado da arte, superando outras classes de algoritmos de aprendizado de máquina\cite{imagenet_2012}
e até mesmo alcançando performances sobre humanas\cite{mnih2015humanlevel}.
Tais avanços foram possíveis devido a três fatores chaves: a viabilização de bases de treino
cada vez maiores o aumento do poder computacional e o desenvolvimento de novas arquiteturas neurais, como redes convolucionais e redes recursivas.

A crescente melhoria de performance dos modelos de aprendizado profundo tem motivado
estudos em áreas onde se é preciso distinguir computadores e humanos. CAPTCHAs \cite{captcha_2003} (do inglês Completely
Automated  Public  Turing  tests  to  tell  Computers  and
Humans Apart) definem uma coleção de técnicas que tem como objetivo bloquear a 
ação de agentes autônomos na rede mundial de computadores. O subconjunto mais conhecido dessas técnicas talvez seja o de captchas baseados em texto\cite{captcha_review_2017}. 
Nesse tipo de desafio, uma imagem contendo uma sequência de caracteres é exibida.
A validação é feita pela comparação entre o texto informado pelo usuário e a resposta
correta. Em trabalhos recentes, foram relatadas acurácias próximos à humana em sequências formadas exclusivamente por números\cite{captcha_break_2013} ou por uma única fonte\cite{captcha_break_2017}. Para o problema geral de 
quebrar captchas baseados em texto, entretanto, modelos de aprendizado profundo ainda mostram
desempenho inferior ao humano. Contudo, pesquisas recentes apontam para avanços claros nos próximos anos\cite{Bursztein2014TheEI}. Em comum, esses modelos possuem a 
necessidade de muito poder computacional e/ou bases de dados extensivas.
O treino dessas redes é tipicamente executado em clusters e/ou sistemas de computação sob demanda, com alto poder de paralelização e utilizando hardware de alto poder de processamento como GPUs e TPUs. Adicionalmente, As bases de treino comumente alcançam alguns terrabytes e envolvem grandes operações de aquisição e/ou geração.


\section{Problema de Pesquisa}

Neste trabalho vamos estudar a viabilidade do treino e validação de redes de aprendizado profundo em computador com pode de processamento mais modesto do que os usualmente utilizados nos melhores resultados encontrados na literatura.
Mas especificamente, investigaremos se é possível construir um modelo de aprendizado profundo
para quebra de captchas de texto em um computador pessoal e ainda alcançar
resultados próximos do estado da arte conhecido na literatura.

\section{Justificativa}

O estado da arte em redes de aprendizado profundo tem aberto portas para aplicações em áreas como processamento de texto \cite{word2vec_2013},
detecção de objetos \cite{Redmon2017YOLO9000BF} e
jogos \cite{mnih2015humanlevel, alphagozero_2017}. Essas aplicações
demandam por uma grande capacidade computacional, o que pode inviabilizar o acesso
a essas novas tecnologias em realidades com orçamento mais baixo ou onde uma prototipação rápida e barata seja necessária. Com este trabalho esperamos demonstrar viabilidade da aplicação de modelos de aprendizado profundo para a quebra de captcha em realidades mais restritivas.

\section{Objetivos}
\textbf{Objetivo Geral:\\}

Testar a viabilidade do uso de modelos de aprendizado profundo 
em um computador pessoal.

\textbf{Objetivos Específicos:\\}
\begin{enumerate}
	\item Investigar técnicas de aprendizado profundo aplicáveis ao problema.
	\item Treinar e/ou validar modelos de aprendizado profundo em um computador pessoal para quebra de captcha.
	\item Disponibilizar os resultados da experimentação de forma pública.
\end{enumerate}

\section{Etapas de Pesquisa}

Durante a execução da pesquisa será realizada uma revisão da literatura as técnicas
utilizadas em aprendizado profundo, com enfoque em  
problemas correlatos à extração de texto de imagens (quebra de captcha).
Experimentar modelos e arquiteturas progressivamente mais complexas, 
sendo o tempo de treino, uso de memória e acurácia na quebra de uma
classe específica de captcha baseado em texto as variáveis de interesse. 


Etapas:\\
\begin{enumerate}
\item Revisão literária.
\item Experimentação.
\item Confecção do TCC.
\item Apresentação do TCC.
\end{enumerate}

\section{Cronograma}

\begin{center}
	\begin{tabular}{|c|c|c|c|}\hline
		\backslashbox{Etapa}{Mês}
		&\makebox[4em]{Maio}
		&\makebox[4em]{Junho}
		&\makebox[4em]{Julho} \\\hline\hline
		
		Revisão bibliográfica & X & X & X \\\hline
		Experimentos 		  &   & X &   \\\hline
		Confecção do TCC      &   & X & X \\\hline
		Apresentação do TCC   &   &   & X \\\hline
	\end{tabular}
\end{center}

\bibliographystyle{unsrt}
\bibliography{bibliography}

\end{document}
