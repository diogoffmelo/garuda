\begin{resumo}[Abstract]
 \begin{otherlanguage*}{english}
  
During the last decade, Deep Neural Networks has been shown to be a powerfull machine learn technique. Generally, to obtain relevant results, these techniques require high computacional power and large volumes of data, which can be a limiting factor on some cases. Neverthless, a careful project of trainig and archtecture may help to reduce these requirements. In the this work we present a comparative approach to the application of deep neural networks to text based CAPTCHAs as a way to cope with these limitations. We studied models that are capable of learn to segment and identify the text content of images, only based on examples. By experimentation of different hiper-parameters and architectures, we were capable to obtain a final model with $96.06\%$ of token prediction accuracy in approximately 3 hours of training in a simple personal computer.
 
   \vspace{\onelineskip}
 
   \noindent
   \textbf{Keywords}: Machine Learning, Deep Learning, CAPTCHA.
 \end{otherlanguage*}
\end{resumo}