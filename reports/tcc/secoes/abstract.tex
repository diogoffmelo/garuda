\begin{resumo}[Abstract]
 \begin{otherlanguage*}{english}
  
  In the this work we present a comparative approach to the application of deep neural networks to text based CAPTCHAs. We studied models that are capable of learn to segment and identify the text content of images, only based on examples. From the experimentation, It was possible to compare the behavior of distinct network architectures and the influence os the learning hiper-parameter on the train dynamics. The network architecture chosen based on the experiments was trained for more epochs, archiving $96.06\%$ of token prediction accuracy in 3 hours and 33 minutes of a training executed on a simples personal computer.
  
 
   \vspace{\onelineskip}
 
   \noindent
   \textbf{Keywords}: Machine Learning, Deep Learning, CAPTCHA.
 \end{otherlanguage*}
\end{resumo}