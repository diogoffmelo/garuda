\chapter{Conclusões}

Neste trabalho propomos uma abordagem comparativa entre diferentes arquiteturas de redes neurais para a solução de CAPTCHAs baseados em texto sem informação humana em um ambiente com capacidade computacional inferior às comumente encontradas na literatura. Mostramos que é possível obter performances próximas ao estado da arte a despeito das limitações. O modelo final escolhido obteve uma acuráciafinal de $96.06\%$ de acerto por token, utilizando menos parâmetros, com requisitos de tempo factíveis e treinado em um mero computador pessoal. Seguem nossas observações para que esse tipo de resultado possa ser reproduzido em outras aplicações.

A experimentação dos hiper-parâmetros é um fator chave na obtenção de bons resultados. Como mostrado pelos experimentos, diferentes configurações do hiper-parâmetro de aprendizado podem levar a dinâmicas de treino substancialmente diferentes. Adicionalmente, diferentes arquiteturas possuem diferentes requisitos. É possível projetar arquiteturas que atendam diferentes especificações (tempo de treino, tamanho, acurácia, etc.) e escolher dentre elas a que melhor se aplica ao problema. Para tal, a análise da dinâmica inicial da rede e o uso de validação cruzada são essenciais. Em particular, a avaliação comparativa da evolução da função custo fornece informação crucial para a escolha e projeto de modelos melhores. Escolhido cuidadosamente uma arquitetura que atenda aos requisitos e os valores ideais para os hiper-parâmetros, é possível realizar seções de treino mais longas, com uma maior exposição do modelo à base de treino e refinar ainda mais os resultados. 

Arquiteturas convolucionais mostraram-se extremamente eficazes para o problema. De fato, esse tipo de camada tem sido aplicada com sucesso em diversos problemas de processamento de imagem. O compartilhamento de parâmetros ao mesmo tempo reduz o tamanho e adiciona maior poder de expressão aos modelos. O \textit{dropout} mostrou-se uma ferramenta eficaz e computacionalmente barata para combater o \textit{overfitting}, melhorando a performance dos modelos estudados sem degradação preceptiva no tempo de computação. A operação de agrupamento com valor fixo apresentou desempenho superior à de \textit{maxout}, tanto no requisito de tempo quanto na acurácia final dos modelos.

\section{Trabalhos futuros}

Segue uma lista de trabalhos futuros que podem enriquecer a discussão iniciapa pelo presente trabalho.

\begin{enumerate}
	\item Explorar mais detalhadamente a influência do número de canais e do tamanho dos núcleos das camadas convolucionais no desempenho das redes, buscando tanto diminuir o número de parâmetros quanto aumentar a acurácia final dos modelos.
	\item Aplicar a mesma abordagem à outros problemas de processamento de imagem, como detecção de objetos, por exemplo, ou outras áreas como predição de sequências temporais.
	\item Estender os modelos os modelos para CAPTCHAs com palavras de sequência variável e/ou diferentes formatos.
	\item Utilizar as mesmas técnicas em CAPTCHAs de texto reais. Em particular, avaliar qual o nível de segurança que esses mecanismos de proteção tem fornecido a instituições no Brasil  
\end{enumerate}
