\chapter{Conclusões e Trabalhos futuros}\label{cap:concusao}

\section{Conclusões}

Neste trabalho propusemos uma abordagem comparativa entre configurações como uma forma efetiva de experimentação para redes neurais profundas. A metodologia proposta se baseia em princípios simples e visa fornecer ao informações importantes que servem de embasamento para a posterior tomada de decisões no projeto de redes neurais. Adicionalmente, os passos propostos podem ser aplicados em ambientes com restrições computacionais, satisfazendo diferentes limitações de recursos. Aplicado ao problema de quebra de CAPTCHAs baseados em texto sem informação humana, mostramos que é possível obter performances próximas ao estado da arte, a despeito das limitações. O modelo final escolhido obteve uma acurácia final de $96.06\%$ de acerto por token, utilizando menos parâmetros, com requisitos de tempo aceitáveis e treinado em um mero computador pessoal. Adicionalmente, a técnica pode ser aplicada de forma iterativa, produzindo resultados mais refinados a cada rodada de experimentos.

\section{Trabalhos futuros}

A seguir apresentamos uma lista não exaustiva de possíveis desdobramentos para o estudo apresentado neste trabalho, separadas por grupos de interesse.

\subsection*{Aperfeiçoamento dos resultados atuais}
De acordo com o método proposto, diversos hiper-parâmetros e detalhes das arquiteturas propostas foram mantidos constantes de forma a permitir uma comparação direta do resultado. Acreditamos que resultados ainda mais expressivos possam ser alcançados através do estudo comparativo dessas variáveis. Dentre elas destacamos o número de canais e o tamanho dos núcleos usados nas camadas. Outra possível melhoria a ser investigada seria a realização de um treinamento mais longo para a rede selecionada.

\subsection*{Validação do método proposto}
Aplicar a mesma abordagem comparativa proposta no presente trabalho à outros problemas de processamento de imagem, como detecção de objetos, por exemplo, ou outras áreas como predição de sequências temporais. Uma aplicação interessante diretamente ligada ao problema de quebra de CAPTCHAs de texto seria estender a nossa definição do problema de forma a permitir tokens com um número variável de caracteres. 

\subsection*{Segurança e CAPTCHAs}
Durante as pesquisas desenvolvidas nesse trabalho notamos que grandes instituições no Brasil se utilizam de CAPTCHAs de texto como medida de segurança. Dada a acurácia alcançada pelo modelo escolhido, propomos como um estudo futuro avaliar o quão eficaz é a segurança desses sistemas através da validação da acurácia obtida por redes neurais profundas aplicadas ai problema. Sugerindo, se possível, melhorias nas técnicas que possam ser aplicadas de forma a tornar o ataque mais difícil.
