\chapter{Introdução}

Algoritmos de aprendizado baseados em neurologia são conhecidos desde meados do século passado \cite{perceptron_58}. Das proposições iniciais até os dias de hoje, essa classe de modelos tem evoluído em complexidade e técnicas de forma contínua,
culminando em um alto poder de expressividade e níveis cada vez mais abstratos de representação (ver \cite{Goodfellow-et-al-2016} ou \cite{jurgenReview2015} para uma breve revisão histórica). Os poucos resultados teóricos disponíveis demonstram que redes neurais possuem um alto poder de generalização, sendo capaz de, sob certas circunstâncias, codificar diversas classes de funções \cite{Barron1993UniversalAB, Andoni2014PolyAprox}. Apesar dos avanços na área, foi apenas recentemente que modelos neurais começaram a redefinir o estado da arte, superando outras classes de algoritmos de aprendizado de máquina \cite{imagenet_2012} e até mesmo alcançando performances sobre humanas \cite{mnih2015humanlevel}. Tais avanços foram possíveis devido a três fatores chaves: a viabilização de bases de dados cada vez maiores, o aumento do poder computacional e o desenvolvimento de novas arquiteturas e técnicas de treino.

A crescente melhoria de performance dos modelos neurais de aprendizado profundo tem motivado estudos em áreas onde se é preciso distinguir computadores e humanos. Dentre essas áreas temos os CAPTCHAs \cite{captcha2003} (do inglês Completely Automated  Public  Turing  tests  to  tell  Computers  and Humans Apart) ou HIPs \cite{lectures2005HIP} (do inglês Human Interaction Proofs), que definem uma coleção de técnicas que tem como objetivo bloquear a ação de agentes autônomos na rede mundial de computadores. Um dos subconjuntos mais conhecidos dessas técnicas talvez seja o de CAPTCHAs baseados em texto \cite{captcha_review_2017}. Nesse tipo de desafio, uma imagem contendo uma sequência de caracteres é exibida e a validação é feita pela comparação entre o texto informado pelo usuário e a resposta correta. Formulado como um problema de aprendizado de máquina, desejamos descobrir de forma automatizada um mapa entre a imagem e o texto codificado. Na versão informada do problema, um ser humano escolhe previamente técnicas de preprocessamento (filtros, segmentação de caracteres, etc.) antes que o aprendizado propriamente dito ocorra. Ajudados por humanos, redes neurais simples e com poucos exemplos conseguem resultados satisfatórios nesse tipo de desafio \cite{lectures2005HIP}. De fato, mesmo técnicas ingênuas como contagem de pixels podem obter bons resultados quando o preprocessamento correto é fornecido \cite{naivecaptcha}. Na versão não informada, entretanto, encontrar mapas imagem-texto de forma automatizada é usualmente muito mais desafiador. Em trabalhos recentes, foram relatados modelos baseados em redes neurais capazes de burlar esse tipo de desafio com acurácias de acerto próximos à humana em sequências sorteadas a partir de um repositório \cite{captcha_break_2013} e modelos com alta eficiência de dados \cite{captcha_break_2017}. Para o problema geral de quebrar CAPTCHAs baseados em texto, entretanto, modelos de aprendizado profundo ainda mostram desempenho inferior ao humano. Contudo, pesquisas recentes apontam para avanços claros nos próximos anos \cite{Bursztein2014TheEI}. Em comum, esses modelos possuem a necessidade de clusters e/ou sistemas de computação sob demanda para treinamento, com hardware de alto poder de processamento e/ou paralelização, como GPUs e TPUs. Adicionalmente, as bases de treinamento necessárias comumente alcançam alguns terabytes e envolvem grandes operações de aquisição e/ou geração.

Neste trabalho propomos uma abordagem comparativa entre diferentes arquiteturas de redes neurais para a solução de CAPTCHAs baseados em texto sem informação humana, nos restringindo, entretanto, à um ambiente com poder computacional reduzido. Pretendemos mostrar que é possível fazer uso dessas técnicas em um mero computador pessoal (na contra-mão dos trabalhos usualmente encontrados na literatura) e ainda obter resultados próximos ao estado da arte. Este trabalho se encontra organizado como segue. No capítulo \ref{cap:captchas} apresentamos uma breve introdução à diferentes tipos de CAPTCHAs, com ênfase em desafios baseados em texto. Sequencialmente, no capítulo \ref{cap:neurais}, arquiteturas e técnicas de projeto e treino de redes neurais comuns na literatura são abordados. Os principais resultados do uso dessas técnicas em CAPTCHAs de texto explorados e nossas considerações iniciais sobre essa aplicação apresentadas. No capítulo \ref{cap:modelagem} uma descrição das arquiteturas dos modelos usados neste estudo é feita em conjunto com uma breve fundamentação para as escolhas. No capítulo \ref{cap:metodologia}, detalhes dos experimentos realizados são formalizados. Por fim, no capítulo \ref{cap:resultados} os resultados dos experimentos são apresentados e analisados e no capítulo \ref{cap:concusao} nossas conclusões e considerações finais apresentadas.
