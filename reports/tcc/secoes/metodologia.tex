\chapter{Metodologia}

\section{Geração dos CAPTCHAS}

foram geradas 30000 imagens usando diferentes efeitos e cores com tokens de comprimento fixo em 5 utilizando-se a biblioteca \cite{simplecaptcha}

um exemplo constitui de um tensor $(200, 50, 3)$ e de um token $(5, 36)$

exemplos normalizados entre 0-1, token codificado one-hot, de modo que 

\begin{equation}
   p(y[i]|x)= 
	\begin{cases}
		1,	& \text{if } y[i]=w_i\\
		0,  & \text{caso contrário.}
	\end{cases}
\end{equation}

Os 30000 exemplos foram separados, de forma aleatória, em dois conjuntos:
o conjunto de treino, $D_{tr}$, e o de validação, $D_{va}$, com 20000 e 10000 exemplos, respectivamente.

\begin{figure}[ht]
	\begin{subfigure}{.5\textwidth}
		\centering
	 	\includegraphics[width=.9\linewidth]{figuras/7103_b26bf.png}
		\caption{b26bf}
	\end{subfigure}
	\begin{subfigure}{.5\textwidth}
		\centering
		\includegraphics[width=.9\linewidth]{figuras/9456_ep8nb.png}
		\caption{ep8nb}
	\end{subfigure}%
	\vspace{.05\linewidth}

	\begin{subfigure}{.5\textwidth}
		\centering
		\includegraphics[width=.9\linewidth]{figuras/21856_b7rw8.png}
		\caption{b7rw8}
	\end{subfigure}
	\begin{subfigure}{.5\textwidth}
		\centering
		\includegraphics[width=.9\linewidth]{figuras/19816_74wf6.png}
		\caption{74wf6}
	\end{subfigure}%
	\vspace{.05\linewidth}

	\begin{subfigure}{.5\textwidth}
		\centering
		\includegraphics[width=.9\linewidth]{figuras/12248_dnyny.png}
		\caption{dnyny}
	\end{subfigure}
	\begin{subfigure}{.5\textwidth}
		\centering
		\includegraphics[width=.9\linewidth]{figuras/8873_g4cxh.png}
		\caption{g4cxh}
	\end{subfigure}%
	\vspace{.05\linewidth}

	\caption{Exemplos de CAPTCHAS gerados e seus respectivos tokens.}
\end{figure}



\section{Grandezas de interesse}

\begin{equation}
	S_i^{(D)} = \sum_{(x,y) \in D} p(y[i]|x) \log \hat{p}(y[i]|x)
\end{equation}

\begin{equation}
	acc_w^{(D)} = \frac{N_w}{|D|}
\end{equation}

\begin{equation}
\hat{p}_i^{(D)} = acc_i^{(D)} = \frac{N_i}{|D|}
\end{equation}

\begin{equation}
\hat{p_w}^{(D)} = \prod_{i} \hat{p}_i^{(D)}
\end{equation}

\begin{equation}
loss_i^{(D)} = \frac{S_i}{|D|}
\end{equation}

\begin{equation}
loss^{(D)} = \sum_{i} loss_i^{(D)}
\end{equation}

$t$ tempo total de execução de uma época (treino + validação)
$\tilde{t}$ tempo de treino em uma época.

\section{Treino e Validação}

Todos as redes foram treinadas em um mesmo computador com pentium core i5, 8gb de ram usando tensorflow.

Redes inicializadas segundo critério de \cite{xavier_init}

etapa de treino consiste em
mini batch: sortear $D_{batch} \subset  D_{tr}$, com $|D_{batch}| = 10$ e minimizar  $S_i$'s nesse conjunto usando com \cite{adam_op} com taxa de aprendizado $l_r$. 

O treino em uma época consiste em repetir $|D_{tr}|/|D_{batch}|$ vezes.

calcular as grandezas de interesse em $D_{tr}$ e $D_{va}$


experimentos realizados com diferentes taxas de aprendizado durante 10 épocas. Limite superior e inferior escolhidos manualmente baseados em melhor desempenho aprendizado rápido para a superior e estável para a inferior. Decaimento linear.

Treinado usando critério de parada: mínimo de 10 épocas, máximo de 50 épocas, custo no conjunto de validação na época atual não maior que o .10 do menor valor encontrado até agora, custo no conjunto de treinamento atual menor que .03 da média dos últimos 5 valores. Inspirado em \cite{lutz_early_stop}
