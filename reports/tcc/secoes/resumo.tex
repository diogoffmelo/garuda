\setlength{\absparsep}{18pt} % ajusta o espaçamento dos parágrafos do resumo
\begin{resumo}
 
Neste trabalho apresentamos uma abordagem comparativa para a aplicação de redes neurais profundas a CAPTCHAs de texto. Estudamos modelos capazes de aprender a segmentar e identificar o texto contido nas imagens baseando-se apenas em exemplos. A partir da experimentação foi possível comparar o comportamento de diferentes arquiteturas e a influência do hiper-parâmetro de aprendizado na dinâmica dp treino. Uma arquitetura escolhida a partir dos experimentos foi treinada por um período maior de épocas, alcançando uma acurácia de $96.06\%$ de acerto por token em apenas 3 horas e 33 minutos de treino executado em um simples computador pessoal.

\vspace{\onelineskip}

\noindent
\textbf{Palavras-chave}: Aprendizado de Maquina, Aprendizado Profundo, CAPTCHA.
\end{resumo}

