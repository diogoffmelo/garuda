\setlength{\absparsep}{18pt} % ajusta o espaçamento dos parágrafos do resumo
\begin{resumo}

Na última década, Redes Neurais Profundas tem se mostrado uma poderosa técnica de aprendizado de máquina. Em geral, essas técnicas demandam alto poder computacional e grandes volumes de dados para obter resultados expressivos, o que pode ser um fator limitante em algumas realidades. Entretanto, o projeto cuidadoso da arquitetura e do treino podem ajudar a reduzir estes requisitos. Neste trabalho apresentamos uma abordagem comparativa para a aplicação de redes neurais profundas à quebra de CAPTCHAs de texto como uma forma de contornar essas limitações. Estudamos modelos capazes de aprender a segmentar e identificar o texto contido em imagens baseando-se apenas em exemplos. A partir da experimentação de diferentes hiper-parâmetros e arquiteturas, fomos capazes de obter um modelo final com acurácia de $96.06\%$ de acerto por token em aproximadamente 3 horas de treino executado em um simples computador pessoal.

\vspace{\onelineskip}

\noindent
\textbf{Palavras-chave}: Aprendizado de Maquina, Aprendizado Profundo, CAPTCHA.
\end{resumo}

