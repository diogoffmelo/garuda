\begin{simbolos}

  \item[$i$, $j$, $k$, $\ldots$] Índices.

  \item[$I$, $J$, $K$, $\ldots$] Conjunto de todos os valores dos índices $i$, $j$, $k$, $\ldots$. $\kappa = (I, J, K, \ldots)$ é uma coleção de índices.

  \item[$\mathbf{x}$] Um vetor.
	 
  \item[$x_i$] Coordenada $i$ do vetor $\mathbf{x}$.
   
  \item[$\mathbf{T^{\kappa}}$] Um tensor com índices na coleção $\kappa$. O tamanho da coleção define o ranque do tensor. Sendo ranque $1$ vetores, $2$ matrizes, etc. Quando definido no contexto, omitiremos $\kappa$.
  
  \item[$T_{i, j, k, \ldots}$] O elemento $i, j, k, \ldots$ do tensor $\mathbf{T}$.
  
  \item[$ \{\ldots\} $] Conjunto de todas as possibilidades de um variável. Usualmente um espaço vetorial.

  \item[$ \langle \ldots \rangle_{D} $] Valor esperado de uma variável no conjunto $D$.

  \item[$| D |$] Número de elementos no conjunto $D$.

  \item[$| \mathbf{T} |_p$] Norma $p$ do tensor $\mathbf{T}$. Quando omitido, $p=2$ (norma euclidiana).

  \item[$\nabla_{\mathbf{T}}$] Operador gradiente com respeito ao tensor $\mathbf{T}$. Isto é, as derivadas em cada elemento de $\mathbf{T}$.
    
  \item[$\Re$] Conjunto dos números reais.
  
  \item[$\Re^{\kappa}$] Espaço dos tensores de ranque $\mathbf{T^{\kappa}}$ sob o conjunto dos números reais. Isto é, $T_{i, j, k, \ldots} \in \Re$.
  
  \item[$ \Re^{\kappa} \lceil \!\!\! \lfloor0,1 \rfloor \!\!\! \rceil $] Compacto dos tensores $\mathbf{T^{\kappa}}$ sob o conjunto $[a,b]$, com $a, b \in \Re$. 
  
  \item[$2^D$] Conjunto de todos os subconjuntos de $D$.
  
  \item[$\mathtt{x}$] Variável aleatória.
  
  \item[$P(\mathtt{x}=a)$] Probabilidade do evento $\mathtt{x}=a$ ocorrer. Quando estiver claro no contexto, utilizamos apenas $P(a)$
  

\end{simbolos}